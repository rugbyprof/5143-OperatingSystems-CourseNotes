\newpage
\mysection{Static Keyword}
\label{sec:static-keyword}


Static Keyword can be used with following,

\begin{itemize}
    \item Static variable in functions
    \item Static Class Objects
    \item Static member Variable in class
    \item Static Methods in class
\end{itemize}

\mysubsection{Static Variables inside Functions}

\begin{itemize}
\tightlist
    \item Static variables when used inside function are initialized only once, and then they hold there value even through function calls.
    \item These static variables are stored on static storage area , not in stack.
\end{itemize}

\begin{minted}[]{c++}
void counter(){
    static int count=0;
    cout << count++;
}

int main({
    for(int i=0;i<5;i++){
        counter();
    }
}
\end{minted}

\textbf{Output:}
\begin{minted}[bgcolor=bg]{text}
0 1 2 3 4
\end{minted}


Let's see the same program's output without using static variable.

\begin{minted}[]{c++}
void counter(){
    int count=0;
    cout << count++;
}

int main({
    for(int i=0;i<5;i++){
        counter();
    }
}
\end{minted}

\textbf{Output:}
\begin{minted}[bgcolor=bg]{text}
0 0 0 0 0
\end{minted}

\begin{itemize}
\tightlist
    \item If we do not use \btic{static} keyword, the variable \btic{count}, is reinitialized every time when \btic{counter()} function is called, and gets destroyed each time when \btic{counter()} functions ends.
    \item But, if we make it \btic{static}, once initialized count will have a scope till the end of main() function and it will carry its value through function calls too.
    \item If you don't initialize a static variable, they are by default initialized to zero.
\end{itemize}

\mysubsection{Static Class Objects}

\begin{itemize}
\tightlist
    \item Static keyword works in the same way for class objects too.
    \item Objects declared static are allocated storage in static storage area, and have scope till the end of program.
    \item Static objects are also initialized using constructors like other normal objects.
    \item Assignment to zero, on using static keyword is only for primitive datatypes, not for user defined datatypes.
\end{itemize}
 
\begin{minted}[]{c++}
class Abc{
    int i;
    public:
    Abc(){
        i=0;
        cout << "constructor";
    }
    ~Abc(){
        cout << "destructor";
    }
};

void f(){
    cout<<"In f()"<<endl;
    static Abc obj;
    cout<<"Exiting f()"<<endl;
}

int main(){
    int x=0;
    if(x==0){
        f();
    }
    cout << "END";
}
\end{minted}

\textbf{Output:}
\begin{minted}[bgcolor=bg]{text}
Start main
Start f()
constructor
End f()
End main
destructor
\end{minted}

You must be thinking, why was the \btic{destructor} not called upon the end of the scope of if condition, where the reference of object \btic{obj} should get destroyed. This is because object was \btic{static}, which has scope till the program's lifetime, hence destructor for this object was called when main() function exits.

\mysubsection{Static Data Member in Class}

\begin{itemize}
\tightlist
    \item \btic{Static} data members of class are those members which are shared by all the objects.
    \item They are allocated a single piece of storage which is the same in every object.
    \itm So, every object has its own copy of all data members \textbf{except static members} which they share with all other objects.
    \item \btic{Static} member variables (data members) are not initialized using a constructor, because these are not dependent on object initialization.
    \item Also, \btic{static members must be initialized explicitly}, always outside the class. If not initialized, Linker will give error.
\end{itemize}

\begin{minted}[]{c++}
class Foo
{
    public:
    static int i;
    Foo()
    {
        // construtor
        cout<<i<<endl;
    };
};

int Foo::i=1;

int main()
{
    Foo obj;
    cout << obj.i;   // prints value of i
}
\end{minted}

\textbf{Output:}
\begin{minted}[bgcolor=bg]{text}
1
\end{minted}

Once the definition for static data member is made, user cannot redefine it. Though, arithmetic operations can be performed on it.

\mysubsection{Static Member Functions}

\begin{itemize}
\tightlist
    \item These functions work for the class as whole rather than for a particular object of a class.
    \item It can be called using an object and the direct member access \btic{.} operator.
    \item However, its more typical to call a static member function by itself, using class name and scope resolution \btic{::} operator.
    \item Functions declared as static, cannot access ordinary data members and member functions, only other statically declared items within the class. 
\end{itemize}

For example:

class Math{
private:
    static int curve;
public:
    static int add(int a, int b){
        return a + b;
    }
    static int sub(int a, int b){
        return a - b;
    }
    static int mul(int a, int b){
        return a * b;
    }
};

int main(){
    cout<<Math::add(2,3)<<endl;
    cout<<Math::sub(2,3)<<endl;
    cout<<Math::mul(2,3)<<endl;
}

\textbf{Output:}
\begin{minted}[bgcolor=bg]{text}
5
-1
6
\end{minted}
